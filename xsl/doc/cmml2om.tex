\documentclass[twoside,11pt]{article}
\usepackage{longtable,array}
\newcommand{\T}{\textstyle}

\setlength{\extrarowheight}{11pt}

\begin{document}
\title{Conversion between MathML and  OpenMath}
\author{D. Carlisle,J. Davenport,M.Dewar,N.Hur,W. Naylor\thanks{Now at the University of Western Ontario.}}
\begin{abstract}
OpenMath and MathML (Content) are both semantic representations of
mathematics. This document shows how all of MathML (Content) can be
translated into OpenMath.
\end{abstract}
\maketitle
\newenvironment{mybox}[1]{
\begin{tabular}{|c|}
\hline
\begin{minipage}[t]{#1}}{
\end{minipage}\\
\hline
\end{tabular}}
\newenvironment{myboxbopen}[1]{
\begin{tabular}{|c|}
\hline
\begin{minipage}[t]{#1}}{
\end{minipage}
\end{tabular}}
\newenvironment{myboxtopen}[1]{
\begin{tabular}{|c|}
\begin{minipage}[t]{#1}}{
\end{minipage}\\
\hline
\end{tabular}}
\section*{Introduction}
This document begins with a table detailing the transformation
required to convert between MathML and OpenMath, using the
MathML CD Group. We then give various illustrations of the more
complex conversions.

The first column gives the MathML element and attribute combination,
the second column gives the equivalent OpenMath construct.
Additional notes and comments are provided where appropriate.
\clearpage

\section*{MathML -- OpenMath}

\begin{longtable}{
        >{\raggedright\let\\=\tabularnewline}p{2in}
        >{\raggedright\let\\=\tabularnewline}p{3in}}
\verb+cn+ or \verb+cn type="real"+&If followed by an explicit floating
point number, then this would normally be represented by an \verb+OMF+
(if expressible in terms of IEEE floats), or possibly via the
\verb+bigfloat1+ CD.\par\noindent
The use of symbolic constants under \verb+cn+ such as \verb+&pi+ is
now deprecated and not supported by OpenMath.\\
\verb+cn type="e-notation"+ (added 18.4.2000 to MathML: syntax is
mantissa, as in real, \verb+<sep/>+, decimal exponent)&As above\\
\verb+cd type="integer"+&Is supported by \verb+OMI+ in base 10 or 16.
Integers expressed in other bases may be converted to a
\verb+<OMS name="based_integer"+ \verb+cd="nums1"/>+ symbol
applied to the digits in an \verb+<OMSTR>+. If they are converted into 
\verb+OMI+s, then the base information is lost.\\
\verb+cn type="rational"+&This is represented by 
\verb+<OMS name="rational"+ \verb+cd="nums1"/>+. Note that MathML allows 1/0 and OpenMath
does not (through this constructor). It appears that MathML allows
decimal points inside \verb+cn type="rational"+, but this is a
curiosity unsupported by OpenMath in this form.\\
\verb+cn type="complex-cartesian"+&This equates to 
\verb+<OMS name="complex_cartesian"+ \verb+cd="complex1"/>+\\
\verb+cn type="complex-polar"+\footnote{This seems to be missing from
appendix C of the Proposed Recommendation, but is clearly present in
Chapter 4.4.}&This equates to 
\verb+<OMS name="complex_polar"+ \verb+cd="complex1"/>+\\
\verb+ci+&This normally corresponds to an \verb+OMV+. However,
\verb+ci type="XXX"+ corresponds to an attribution from
the \verb+mathmltypes+ CD on the \verb+OMV+.\\
\verb+csymbol+&If the URL points to an OpenMath CD, then this refers
to the symbol of that name in the CD.\\
\verb+apply+&Corresponds to \verb+OMA+.\\
\verb+reln+ (now deprecated in MathML)&Corresponds to \verb+OMA+.\\
\verb+fn+ (now deprecated in MathML)&Is ignored in OpenMath.\\
\verb+interval+&According to the type attribute (which defaults to
\verb+closed+) corresponds to a symbol in the \verb+interval1+ CD.\\
\verb+inverse+&Corresponds, with all its (lack of) semantics, to
\verb+<OMS name="inverse"+ \verb+cd="fns1"/>+.\\
\verb+sep+&Handled by its parent constructor.\\
\verb+condition+ inside \verb+int+&This becomes the first argument to
\verb+<OMS name="defint" cd="calculus1"/>+.\\
\verb+int+ with a general range&This may be performed by using
\verb+<OMS name="defint" cd="calculus1"/>+ where the range argument is
an \verb+OMV+. See illustration \ref{general range}.\\
\verb+condition+ inside \verb+sum+ and \verb+product+&This becomes the
first argument to \verb+<OMS name="sum" cd="arith1"/>+.\\
\verb+condition+ inside \verb+limit+&This should be a \verb+tendsto+
construct which gives the arguments of the OpenMath construction. See
illustration \ref{limit}.\\
\verb+condition+ inside \verb+min+ and \verb+max+&Becomes a
\verb+<OMS name="suchthat" cd="set1"/>+.\\
\verb+condition+ inside \verb+forall+ and \verb+exists+&The condition
is moved inside the body (normally as \verb+implies+ for \verb+forall+, 
but \verb+and+ for \verb+exists+): see illustration \ref{forall}.\\
\verb+condition+ inside \verb+set+&Becomes a
\verb+<OMS name="suchthat"+ \verb+cd="set1"/>+.\\ 
\verb+condition+ inside \verb+list+&Becomes a
\verb+<OMS name="suchthat"+ \verb+cd="list1"/>+.\\
\verb+declare type="XXX"+&This can be supported by replacing all
occurrences of the name by the name attributed by the type from the
\verb+mathmltypes+ CD. Other instances of \verb+declare+ are not
supported in OpenMath.\\
\verb+lambda+&Corresponds to \verb+OMBIND+ with a first child of
\verb+<OMS name="lambda"+ \verb+cd="fns1"/>+. The \verb+bvar+ elements under
the MathML \verb+lambda+ become children of the OpenMath
\verb+OMBIND+.\\
\verb+compose+&Corresponds to \verb+<OMS name="left_compose"+ 
\verb+cd="fns1"/>+.\\
\verb+ident+&Corresponds to \verb+<OMS name="identity"+ \verb+cd="fns1"/>+.\\
\verb+domain+&Corresponds to \verb+<OMS name="domain"+ \verb+cd="fns1"/>+.\\
\verb+domainofapplication+&Corresponds to 
\verb+<OMS name="domainofapplication"+ \verb+cd="fns1"/>+. However,
when used inside \verb+int+ to give a general range of integration, it
becomes an argument of \verb+defint+, without any \verb+domainofapplication+
%(see example \ref{DOA})%
.\\
\verb+codomain+&Corresponds to \verb+<OMS name="image"+ \verb+cd="fns1"/>+.\\
\verb+image+&Corresponds to \verb+<OMS name="range"+ \verb+cd="fns1"/>+.\\
\verb+quotient+&Corresponds to \verb+<OMS name="quotient"+ 
\verb+cd="integer1"/>+. The \verb+type+ qualifier, if not \verb+integer+, is 
not supported in OpenMath.\\
\verb+factorial+&Corresponds to \verb+<OMS name="factorial"+
\verb+cd="integer1"/>+. The \verb+type+ qualifier, if not \verb+integer+, 
is not supported in OpenMath.\\
\verb+divide+&Corresponds to \verb+<OMS name="divide"+
\verb+cd="arith1"/>+. The \verb+type+ qualifier is ignored.\\
\verb+max+ and \verb+min+&If these MathML constructors are applied to
something other than a \verb+condition+, this
becomes an OpenMath application of \verb+<OMS name="max"+ \verb+cd="minmax1"/>+
to a set built with \verb+<OMS name="set"+ \verb+cd="set1"/>+. 
If applied to a \verb+condition+, this becomes an
application of \verb+<OMS name="max"+ \verb+cd="minmax1"/>+ 
to a set built with \verb+<OMS name="suchthat"+ \verb+cd="set1"/>+.\\
\verb+minus+&Corresponds to \verb+<OMS name="minus"+ \verb+cd="arith1"/>+ or
\verb+<OMS name="unary_minus"+ \verb+cd="arith1"/>+, depending on the
arity. However, applications to objects of type \verb+set+ or
\verb+multiset+, as envisaged in section C.2.3.6 of the MathML2
standard, should be translated into \verb+<OMS name="setdiff"+
\verb+cd="set1"/>+ (or \verb+cd="multiset1"/>+).\\
\verb+plus+&Corresponds to \verb+<OMS name="plus"+ \verb+cd="arith1"/>+. 
However, \verb+<apply><plus/></apply>+ corresponds to
\verb+<OMS name="zero"+ \verb+cd="alg1"/>+. Furthermore, applications to 
objects of type \verb+set+ or \verb+multiset+, as envisaged in section C.2.3.7
of the MathML2 standard, should be translated into \verb+<OMS name="union"+
\verb+cd="set1"/>+ (or \verb+cd="multiset1"/>+).\\
\verb+power+&This corresponds to \verb+<OMS name="power"+ \verb+cd="arith1"/>+.\\
\verb+real+ and \verb+imaginary+&These correspond to
\verb+<OMS name="real"+ \verb+cd="complex1"/>+ (and \verb+imaginary+).\\
\verb+rem+&Corresponds to \verb+<OMS name="remainder"+ 
\verb+cd="integer1"/>+. The \verb+type+ qualifier, if not \verb+integer+, is 
not supported in OpenMath.\\
\verb+times+&This corresponds to \verb+<OMS name="times"+ \verb+cd="arith1"/>+.\\
\verb+root+&This corresponds to \verb+<OMS name="root"+
\verb+cd="arith1"/>+. However, note that the OpenMath semantics are
those of \verb+type="principal_branch"+ (misspelled as
\verb+type="principle_branch"+ in Appendix C of the Proposed
Recommendation for MathML2), and the default type of \verb+real+ is
not directly supported in OpenMath.\\
\verb+gcd+&This corresponds to
\verb+<OMS name="gcd"+ \verb+cd="arith1"/>+. However,
\verb+<apply><gcd/></apply>+ corresponds to 
\verb+<OMS name="one"+ \verb+cd="alg1"/>+ (logically, it should be
$0$, but this is the MathML statement, in C.2.3.16).\\
\verb+and+&This corresponds to \verb+<OMS name="and"+ \verb+cd="logic1"/>+.\\
\verb+or+&This corresponds to \verb+<OMS name="or"+ \verb+cd="logic1"/>+.\\
\verb+xor+&This corresponds to
\verb+<OMS name="xor"+ \verb+cd="logic1"/>+. MathML has now adopted the
OpenMath semantics for the non-binary case.\\
\verb+not+&This corresponds to \verb+<OMS name="not"+ \verb+cd="logic1"/>+.\\
\verb+implies+&This corresponds to \verb+<OMS name="implies"+ \verb+cd="logic1"/>+.\\
\verb+forall+&Corresponds to \verb+OMBIND+ with a first child of
\verb+<OMS name="forall"+ \verb+cd="quant1"/>+. The \verb+bvar+ elements under
the MathML \verb+forall+ become children of the OpenMath
\verb+OMBIND+. Any \verb+condition+ has to be moved inside the
body, as in illustration \ref{forall}.\\
\verb+exists+&Corresponds to \verb+OMBIND+ with a first child of
\verb+<OMS name="exists"+ \verb+cd="quant1"/>+. The \verb+bvar+ elements under
the MathML \verb+exists+ become children of the OpenMath
\verb+OMBIND+. Any \verb+condition+ has to be moved inside the
body, as in illustration \ref{forall}.\\
\verb+abs+&This corresponds to \verb+<OMS name="abs"+ \verb+cd="arith1"/>+.\\
\verb+conjugate+&This corresponds to
\verb+<OMS name="conjugate"+ \verb+cd="complex1"/>+.\\
\verb+arg+&This corresponds to \verb+<OMS name="argument"+
\verb+cd="complex1"/>+.\\
\verb+lcm+&This corresponds to
\verb+<OMS name="lcm"+ \verb+cd="arith1"/>+. However,
\verb+<apply><lcm/></apply>+ corresponds to 
\verb+<OMS name="zero"+ \verb+cd="alg1"/>+ (logically, it should be
$1$, but this is the MathML statement, in C.2.3.25).\\
\verb+floor+&This corresponds to \verb+<OMS name="floor"+ 
\verb+cd="rounding1"/>+.\\
\verb+ceiling+&This corresponds to \verb+<OMS name="ceiling"+ 
\verb+cd="rounding1"/>+.\\
\verb+eq+&This corresponds to 
\verb+<OMS name="eq"+ \verb+cd="relations1"/>+. However, the MathML construct 
is $n$-ary, but the OpenMath construct is binary. See illustration
\ref{eq-unfold}.\\
\verb+neq+&This corresponds to
\verb+<OMS name="neq"+ \verb+cd="relations1"/>.+\\
\verb+gt+, \verb+lt+, \verb+geq+ and \verb+leq+.&These correspond to
\verb+<OMS name="gt"+ \verb+cd="relations1"/>+ etc. However, the MathML construct 
is $n$-ary, but the OpenMath construct is binary. See illustration
\ref{gt-unfold}.\\
\verb+equivalent+\footnote{Chapter 4 and Appendix C of the Proposed
Recommendation differ here: we follow Chapter 4.}&This corresponds to
\verb+<OMS name="equivalent"+ \verb+cd="logic1"/>+.\\
\verb+approx+&This corresponds to
\verb+<OMS name="approx"+ \verb+cd="relation1"/>+. However, the MathML 
construct is $n$-ary, but the OpenMath construct is binary. See illustration
\ref{eq-unfold}.\\
\verb+factorof+&This corresponds to \verb+<OMS name="factorof"+ 
\verb+cd="integer1"/>+.\\
\verb+int+&This corresponds either to 
\verb+<OMS name="int"+ \verb+cd="calculus1"/>+ or
\verb+<OMS name="defint"+ \verb+cd="calculus1"/>+. See illustration
\ref{int}.\\
\verb+diff+&This corresponds to \verb+<OMS name="diff"+ \verb+cd="calculus1"/>+.
See illustration \ref{diff}.\\
\verb+partialdiff+&This corresponds to 
\verb+<OMS name="partialdiff"+ \verb+cd="calculus1"/>+.
See illustration \ref{partialdiff}.\\
\verb+lowlimit+, \verb+uplimit+, \verb+bvar+ and \verb+degree+ are all
handled by their parents.\\
\verb+divergence+&This corresponds to 
\verb+<OMS name="divergence"+ \verb+cd="veccalc1"/>+.\\
\verb+gradient+&This corresponds to 
\verb+<OMS name="grad"+ \verb+cd="veccalc1"/>+.\\
\verb+curl+&This corresponds to 
\verb+<OMS name="curl"+ \verb+cd="veccalc1"/>+. There is not a direct
OpenMath correspondence for the MathML form that takes three variables: a
$\lambda$-expression has to be built.\\
\verb+laplacian+&This corresponds to 
\verb+<OMS name="Laplacian"+ \verb+cd="veccalc1"/>+.\\
\verb+set+&The construct with $n$ explicit arguments corresponds to
\verb+<OMS name="set"+ \verb+cd="set1"/>+, or
\verb+<OMS name="multiset"+ \verb+cd="multiset1"/>+ if \verb+type=multiset+
is specified. The form with a \verb+condition+
translates to a use of \verb+<OMS name="suchthat"+ \verb+cd="set1"/>+. See
illustration \ref{set-suchthat}. There is no OpenMath equivalent if
\verb+type=multiset+ is specified with a \verb+condition+.\\
\verb+list+&The construct with $n$ explicit arguments corresponds to
\verb+<OMS name="list"+ \verb+cd="list1"/>+. The form with a \verb+condition+
translates to a use of \verb+<OMS name="make_list"+ \verb+cd="list1"/>+. See
illustration \ref{list-condition}.\\
\verb+union+&This corresponds to \verb+<OMS name="union"+ \verb+cd="set1"/>+, or
\verb+<OMS name="union"+ \verb+cd="multiset1"/>+. In the absence of a
\verb+definitionURL+, it defaults to \verb+set1+.\\
\verb+intersect+&This corresponds to
\verb+<OMS name="intersect"+ \verb+cd="set1"/>+, or
\verb+<OMS name="intersect"+ \verb+cd="multiset1"/>+. In the absence of a
\verb+definitionURL+, it defaults to \verb+set1+.\\
\verb+notin+, \verb+in+, \verb+subset+, \verb+prsubset+,
\verb+notsubset+ and \verb+notprsubset+&These correspond to
\verb+<OMS name="notin"+ \verb+cd="set1"/>+, or 
\verb+<OMS name="notin"+ \verb+cd="multiset1"/>+ etc. In the absence of a
\verb+definitionURL+, these default to \verb+set1+.\\
\verb+setdiff+&This corresponds to \verb+<OMS name="setdiff"+ 
\verb+cd="set1"/>+, or \verb+<OMS name="setdiff"+ \verb+cd="multiset1"/>+. In 
the absence of a \verb+definitionURL+, it defaults to \verb+set1+.\\
\verb+card+&This corresponds to \verb+<OMS name="size"+ \verb+cd="set1"/>+, or
\verb+<OMS name="size"+ \verb+cd="multiset1"/>+. In 
the absence of a \verb+definitionURL+, it defaults to \verb+set1+.\\
\verb+cartesianproduct+&This corresponds to
\verb+<OMS name="cartesian_product"+ \verb+cd="set1"/>+, or
\verb+<OMS name="cartesian_product"+ \verb+cd="multiset1"/>+. In 
the absence of a \verb+definitionURL+, it defaults to \verb+set1+.\\
\verb+sum+ and \verb+product+&These correspond to
\verb+<OMS name="sum"+ \verb+cd="arith1"/>+ etc. The \verb+uplimit+
etc. correspond to the first argument of the OpenMath construction.\\ 
\verb+limit+&This corresponds to
\verb+<OMS name="limit"+ \verb+cd="limit1"/>+. The MathML \verb+bvar+
corresponds to the variable in the $\lambda$-expression that is the
third argument of the OpenMath construction. The \verb+lowlimit+
corresponds to the first argument of the OpenMath limit (which is otherwise
extracted from the \verb+condition+), and the second argument defaults
to \verb+<OMS name="both_sides"+ \verb+cd="limit1"/>+, if not specified by
the \verb+type+ attribute of the MathML \verb+tendsto+. See
illustration \ref{limit}.\\
\verb+tendsto+&This is handled by the parent \verb+limit+.\\
\verb+exp+, \verb+ln+&These correspond to
\verb+<OMS name="exp"+ \verb+cd="transc1"/>+ etc.\\
\verb+log+&The two-argument form corresponds to
\verb+<OMS name="log"+ \verb+cd="transc1"/>+. The one-argument form has to have an
explicit base of 10 supplied (the first argument of the OpenMath form).\\
\verb+logbase+&Subsumed within \verb+log+.\\
\verb+cos+ etc. (24 of them)&These correspond to the functions in the
\verb+transc1+ CD.\\
\verb+sdev+, \verb+variance+, \verb+median+, \verb+mode+ and
\verb+mean+&These correspond to either
\verb+<OMS name="sdev"+ \verb+cd="s-data1"/>+ etc. or
\verb+<OMS name="sdev"+ \verb+cd="s-dist1"/>+ etc., depending on whether the
arguments are explicit data or a distribution. The unary case in
MathML is that of a distribution.\\
\verb+moment+&The definition of \verb+moment+ in MathML is undergoing
debate. The (now deprecated) \verb+degree+ qualifier in MathML becomes the 
first argument of the OpenMath constructor
\verb+<OMS name="moment"+ \verb+cd="s-dist1"/>+ (and possibly 
\verb+<OMS name="moment"+ \verb+cd="s-data1"/>+), the second argument
specifies the point about which the moment is to be taken.\\
\verb+momentabout+&Subsumed in \verb+moment+.\\
\verb+vector+&This corresponds to
\verb+<OMS name="vector"+ \verb+cd="linalg3"/>+ if the type is
\verb+column+ (the default), otherwise to \verb+<OMS name="vector"+ 
\verb+cd="linalg2"/>+.\\
\verb+matrix+ and \verb+matrixrow+&These correspond to 
\verb+<OMS name="matrix"+ \verb+cd="linalg2"/>+ etc.\\
\verb+determinant+ and \verb+transpose+&Correspond to 
\verb+<OMS name="determinant"+ \verb+cd="linalg1"/>+ etc.\\
\verb+selector+&Corresponds to 
\verb+<OMS name="vector_selector"+ \verb+cd="linalg1"/>+ or
\verb+<OMS name="matrix_selector"+ \verb+cd="linalg1"/>+, depending on the
MathML arity.\\
\verb+vectorproduct+, \verb+scalarproduct+ and
\verb+outerproduct+&These correspond to
\verb+<OMS name="vectorproduct"+ \verb+cd="linalg1"/>+ etc.\\
\verb+integers+&This corresponds to
\verb+<OMS name="Z"+ \verb+cd="setname1"/>+.\\
\verb+reals+&This corresponds to
\verb+<OMS name="R"+ \verb+cd="setname1"/>+.\\
\verb+rationals+&This corresponds to
\verb+<OMS name="Q"+ \verb+cd="setname1"/>+.\\
\verb+naturalnumbers+&This corresponds to
\verb+<OMS name="N"+ \verb+cd="setname1"/>+.\\
\verb+complexes+&This corresponds to
\verb+<OMS name="C"+ \verb+cd="setname1"/>+.\\
\verb+primes+&This corresponds to
\verb+<OMS name="P"+ \verb+cd="setname1"/>+.\\
\verb+exponentiale+&This corresponds to
\verb+<OMS name="e"+ \verb+cd="nums1"/>+.\\
\verb+imaginaryi+&This corresponds to
\verb+<OMS name="i"+ \verb+cd="nums1"/>+.\\
\verb+notanumber+&This corresponds to
\verb+<OMS name="NaN"+ \verb+cd="nums1"/>+.\\
\verb+true+&This corresponds to
\verb+<OMS name="true"+ \verb+cd="logic1"/>+.\\
\verb+false+&This corresponds to
\verb+<OMS name="false"+ \verb+cd="logic1"/>+.\\
\verb+emptyset+&This corresponds to
\verb+<OMS name="emptyset"+ \verb+cd="set1"/>+, or possibly 
\verb+<OMS name="emptyset"+ \verb+cd="multiset1"/>+ for the multiset form.\\
\verb+pi+&This corresponds to
\verb+<OMS name="pi"+ \verb+cd="nums1"/>+.\\
\verb+eulergamma+&This corresponds to
\verb+<OMS name="gamma"+ \verb+cd="nums1"/>+.\\
\verb+infinity+&This corresponds to
\verb+<OMS name="infinity"+ \verb+cd="nums1"/>+. The attribute
\verb+type=complex+ is handled by attributing from the
\verb+mathmltypes+ CD.\\
\end{longtable}

Note that OpenMath does not support MathML's \verb+definitionURL+
unless the URL points to an OpenMath CD.\\
\begin{center}
\begin{mybox}{4.5in}
The MathML version:
\begin{verbatim}<apply>
  <eq/>
  <apply>
    <limit/>
    <bvar> <ci> x </ci> </bvar>
    <condition>
      <apply>
        <tendsto type="above"/>
        <ci> x </ci> <cn> 0 </cn>
      </apply>
    </condition>
    <apply>
      <divide/> <cn> 1 </cn> <ci> x </ci>
    </apply>
  </apply>
  <infinity/>
</apply>
\end{verbatim}

The OpenMath version:
\begin{verbatim}
<OMOBJ>
  <OMA>
    <OMS cd="relation1" name="eq"/>
    <OMA>
      <OMS cd="limit1" name="limit"/>
      <OMI> 0 </OMI>
      <OMS cd="limit1" name="above"/>
      <OMBIND>
        <OMS cd="funs1" name="lambda"/>
        <OMBVAR> <OMV name="x"/> </OMBVAR>
        <OMA>
          <OMS cd="arith1" name="divide"/>
          <OMI> 1 </OMI>
          <OMV name="x"/>
        </OMA>
      </OMBIND>
    </OMA>
    <OMS cd="nums1" name="infinity"/>
  </OMA>
</OMOBJ>
\end{verbatim}
\end{mybox}
\end{center}
\begin{equation}\label{limit}
\hbox{An example of a limit construct to illustrate the condition
{\em tends to from above}.}
\end{equation}

\begin{center}
\begin{myboxbopen}{4.5in}
The MathML version:
\begin{verbatim}
<apply>
  <int/>
  <bvar>
     <ci> x </ci>
  </bvar>
  <lowlimit>
     <bvar>
       <ci> a </ci>
     </bvar>
  </lowlimit>
  <uplimit>
     <bvar>
       <ci> b </ci>
     </bvar>
  </uplimit>
  <apply>
     <ci> f </ci>
     <ci> x </ci>
  </apply>
</apply>
\end{verbatim}

The OpenMath version:
\begin{verbatim}
<OMA>
  <OMS cd="calculus1" name="defint"/>
  <OMA>
    <OMS cd="interval1" name="interval"/>
    <OMV name="a"/>
    <OMV name="b"/>
  </OMA>
  <OMBIND>
    <OMS cd="fns1" name="lambda"/>
    <OMBVAR>
      <OMV name="x"/>
    </OMBVAR>
    <OMA>
      <OMV name="f"/>
      <OMV name="x"/>
    </OMA>
  </OMBIND>
</OMA>
\end{verbatim}
\end{myboxbopen}\end{center}
\begin{center}
\begin{myboxtopen}{4.5in}
However we could specify effectively the same integral without
labelling the upper and lower bounds of the range of integration in the
following manner:
\begin{verbatim}
<OMA>
  <OMS cd="calculus1" name="defint"/>
  <OMV name="R"/>
  <OMBIND>
    <OMS cd="fns1" name="lambda"/>
    <OMBVAR>
      <OMV name="x"/>
    </OMBVAR>
    <OMA>
      <OMV name="f"/>
      <OMV name="x"/>
    </OMA>
  </OMBIND>
</OMA>
\end{verbatim}
\end{myboxtopen}
\end{center}
\begin{equation}\label{general range}
\hbox{An example to show the specification of a general range of integration.}
\end{equation}

\begin{center}
\begin{myboxbopen}{4.5in}
The MathML version:
\begin{verbatim}
<apply>
  <forall/>
  <bvar> <ci> x </ci> </bvar>
  <condition>
    <apply>
      <in/> <ci> x </ci>
      <naturalnumbers/>
    </apply>
  </condition>
  <apply>
    <in/>
    <apply>
      <plus/>
      <ci> x </ci> <cn> 1 </cn>
    </apply>
    <naturalnumbers/>
  </apply>
</apply>
\end{verbatim}
\end{myboxbopen}\end{center}
\begin{center}
\begin{myboxtopen}{4.5in}
The OpenMath version:
\begin{verbatim}
<OMOBJ>
  <OMBIND>
    <OMS cd="quant1" name="forall"/>
    <OMBVAR>
      <OMV name="x"/>
    </OMBVAR>
    <OMA>
      <OMS cd="logic1" name="implies"/>
      <OMA>
        <OMS cd="set1" name="in"/> <OMV name="x"/>
        <OMS cd="setname1" name="N"/>
      </OMA>
      <OMA>
        <OMS cd="set1" name="in"/>
        <OMA>
          <OMS cd="arith1" name="plus"/>
          <OMV name="x"/> <OMI> 1 </OMI>
        </OMA>
        <OMS cd="setname1" name="N"/>
      </OMA>
    </OMA>
  </OMBIND>
</OMOBJ>
\end{verbatim}
\end{myboxtopen}
\end{center}
\begin{equation}
\hbox{An example: for all x in the natural numbers, x+1 is a natural number.}
\label{forall}
\end{equation}

\begin{center}
\begin{mybox}{4.5in}

The OpenMath equivalent to the MathML for a=b=c, which is:
\begin{verbatim}
<apply>
  <eq/>
  <ci> a </ci>
  <ci> b </ci>
  <ci> c </ci>
</apply>
\end{verbatim}
is
\begin{verbatim}
<OMOBJ>
  <OMA>
    <OMS cd="logic1" name="and"/>
    <OMA>
      <OMS cd="relation1" name="eq"/>
      <OMV name="a"/>
      <OMV name="b"/>
    </OMA>
    <OMA>
      <OMS cd="relation1" name="eq"/>
      <OMV name="b"/>
      <OMV name="c"/>
    </OMA>
  </OMA>
</OMOBJ>

\end{verbatim}
\end{mybox}
\end{center}
\begin{center}
Illustration of the {\em unfolding} of the MathML nary equality operator
to enable the use of the binary OpenMath {\bf eq} noperator by using the
OpenMath {\bf and} operator.
\end{center}
\begin{equation}\label{eq-unfold}
\end{equation}

\begin{center}
\begin{mybox}{4.5in}
  
The OpenMath equivalent to the MathML for a$>$b$>$c, which is:
\begin{verbatim}
<apply>
  <gt/>
  <ci> a </ci>
  <ci> b </ci>
  <ci> c </ci>
</apply>
\end{verbatim}
is
\begin{verbatim}
<OMOBJ>
  <OMA>
    <OMS cd="logic1" name="and"/>
    <OMA>
      <OMS cd="relation1" name="gt"/>
      <OMV name="a"/>
      <OMV name="b"/>
    </OMA>
    <OMA>
      <OMS cd="relation1" name="gt"/>
      <OMV name="b"/>
      <OMV name="c"/>
    </OMA>
  </OMA>
</OMOBJ>

\end{verbatim}
\end{mybox}
\end{center}
\begin{center}
Illustration of the {\em unfolding} of the MathML nary greater than operator
to enable the use of the binary OpenMath {\bf gt} operator by using the
OpenMath {\bf and} operator.
\end{center}
\begin{equation}\label{gt-unfold}
\end{equation}

\begin{center}
\begin{mybox}{4.5in}
The MathML version:
\begin{verbatim}
<apply>
  <int/>
  <bvar>
    <ci> x </ci>
  </bvar>
  <apply>
    <sin/>
    <ci> x </ci>
  </apply>
</apply>
\end{verbatim}

The OpenMath version:
\begin{verbatim}
<OMOBJ>
  <OMA>
    <OMS cd="calculus1" name="int"/>
    <OMBIND>
      <OMS cd="fns1" name="lambda"/>
      <OMBVAR>
        <OMV name="x"/>
      </OMBVAR>
      <OMA>
        <OMS cd="transc1" name="sin"/>
        <OMV name="x"/>
      </OMA>
    </OMBIND>
  </OMA>
</OMOBJ>

\end{verbatim}
\end{mybox}
\end{center}
\begin{equation}\label{int}
\hbox{Representation of the integral of the function $\sin(x)$ with respect to $x$.}
\end{equation}

\begin{center}
\begin{mybox}{4.5in}
The MathML version:
\begin{verbatim}
<apply>
  <diff/>
  <bvar>
    <ci> x </ci>
  </bvar>
  <apply>
    <sin/>
    <ci> x </ci>
  </apply>
</apply>
\end{verbatim}

The OpenMath version:
\begin{verbatim}
<OMOBJ>
  <OMA>
    <OMS cd="calculus1" name="diff"/>
    <OMBIND>
      <OMS cd="fns1" name="lambda"/>
      <OMBVAR>
        <OMV name="x"/>
      </OMBVAR>
      <OMA>
        <OMS cd="transc1" name="sin"/>
        <OMV name="x"/>
      </OMA>
    </OMBIND>
  </OMA>
</OMOBJ>
\end{verbatim}
\end{mybox}
\end{center}
\begin{center}
Representation of the differential of the function $\sin(x)$ with
respect to $x$.
\end{center}
\begin{equation}\label{diff}
\end{equation}

\begin{center}
\begin{mybox}{4.5in}
The MathML version:
\begin{verbatim}
<apply>
  <partialdiff/>
  <bvar> x </bvar>
  <bvar> z </bvar>
  <cn> 3 </cn>
  <apply>
    <times/>
    <ci> x </ci><ci> y </ci><ci> z </ci>
  </apply>
</apply>
\end{verbatim}

The OpenMath version:
\begin{verbatim}
<OMOBJ>
  <OMA>
    <OMS cd="calculus1" name="partialdiff"/>
    <OMA>
      <OMS cd="list1" name="list"/>
      <OMI> 1 </OMI>
      <OMI> 0 </OMI>
      <OMI> 3 </OMI>
    </OMA>
    <OMBIND>
      <OMS cd="fns1" name="lambda"/>
      <OMBVAR>
        <OMV name="x"/>
        <OMV name="y"/>
        <OMV name="z"/>
      </OMBVAR>
      <OMA>
        <OMS cd="arith1" name="times"/>
        <OMV name="x"/>
        <OMV name="y"/>
        <OMV name="z"/>
      </OMA>
    </OMBIND>
  </OMA>
</OMOBJ>
\end{verbatim}
\end{mybox}
\end{center}
\begin{equation}\label{partialdiff}
\hbox{Representation of the partial-differential expression
$\frac{\T \partial^4}{\T \partial x\partial z^3}(xyz)$.}
\end{equation}


\begin{center}
\begin{mybox}{4.5in}
The MathML version:
\begin{verbatim}
<set>
  <bvar> x </bvar>
  <condition>
    <apply>
      <in/>
      <apply>
        <divide/>
        <ci> x </ci>
        <cn> 2 </cn>
      </apply>
      <integers/>
    </apply>
  </condition>
</set>
\end{verbatim}

The OpenMath version:
\begin{verbatim}
<OMOBJ>
  <OMA>
    <OMS cd="set1" name="suchthat"/>
    <OMS cd="setname1" name="Z"/>
    <OMBIND>
      <OMS cd="funs1" name="lambda"/>
      <OMBVAR> <OMV name="x"/> </OMBVAR>
      <OMA>
        <OMS cd="set1" name="in"/>
        <OMA>
          <OMS cd="arith1" name="divide"/>
          <OMV name="x"/>
          <OMI> 2 </OMI>
        </OMA>
        <OMS cd="setname1" name="Z"/>
      </OMA>
    </OMBIND>
  </OMA>
</OMOBJ>
\end{verbatim}
\end{mybox}
\end{center}
\begin{center}
Construct the set of even integers, using the suchthat
constructor. Note that making the integers ({\bf Z}) the second
argument of suchthat is a deep semantic transformation, as implied in
the MathML version.
\end{center}
\begin{equation}\label{set-suchthat}
\end{equation}

\begin{center}
\begin{myboxbopen}{4.5in}
The MathML version:
\begin{verbatim}
<list>
  <bvar> x </bvar>
<condition>
  <apply>
    <and/>
    <apply>
      <geq/>
      <ci> x </ci> <cn> 0 </cn>
    </apply>
    <apply>
      <lt/>
      <ci> x </ci> <cn> 100 </cn>
    </apply>
    <apply>
      <in/>
      <apply>
        <divide/>
        <ci> x </ci> <cn> 2 </cn>
      </apply>
      <integers/>
    </apply>
  </apply>
</condition>
</list>
\end{verbatim}

\end{myboxbopen}
\end{center}
\begin{center}
\begin{myboxtopen}{4.5in}
The OpenMath version:
\begin{verbatim}
<OMOBJ>
  <OMA>
    <OMS cd="list1" name="suchthat"/>
    <OMS cd="nums1" name="Z"/>
    <OMBIND>
      <OMS cd="funs1" name="lambda"/>
      <OMBVAR>
        <OMV name="x"/>
      </OMBVAR>
      <OMA>
        <OMS cd="logic1" name="and"/>
        <OMA>
          <OMS cd="relation1" name="geq"/>
          <OMV name="x"/>
          <OMS cd="alg1" name="zero"/>
        </OMA>
        <OMA>
          <OMS cd="relation1" name="lt"/>
          <OMV name="x"/>
          <OMI> 100 </OMI>
        </OMA>
        <OMA>
          <OMS cd="set1" name="in"/>
          <OMA>
            <OMS cd="arith1" name="divide"/>
            <OMV name="x"/> <OMI> 2 </OMI>
          </OMA>
          <OMS cd="funs1" name="Z"/>
        </OMA>
      </OMA>
    </OMBIND>
  </OMA>
</OMOBJ>
\end{verbatim}
\end{myboxtopen}
\end{center}
\begin{center}
This example shows how to construct the list of even positive integers less
than 100, using the suchthat constructor.
\end{center}
\begin{equation}\label{list-condition}
\end{equation}
\end{document}

%%% Local Variables: 
%%% mode: latex
%%% TeX-master: t
%%% End: 
